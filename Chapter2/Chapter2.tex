\chapter{Introduction to Algorithmic Design}

\section{Computation and Computerisation}

Today, in almost all engineering and architectural fields, computers are used heavily and almost
exclusively for any type of work. Some of the utilisation of computers in architecture such as 3D
NURBS or Mesh modelling programs, or any mouse manipulated 3D forms programmes, are often thought to
be computational. Although they essentially use numerical computation to display or 'translate'
\cite{terzidis06} different forms, ultimately it is the user who willingly manipulates the models.
This is a confusion between computation and computerisation. 

\paragraph{Computerisation}could be defined as using the computer as a tool instead of traditional
tools or media. One could use ink and paper to write a letter, or use a word processor instead. The
presentation is probably better, the work is done faster, but the content would not change on
account of this tool swap. The same notion basically applies to CAD programmes, which are merely
tools to draw shapes in a manner which essentially replaces pencil and paper, because ultimately the
user is in complete control.

\paragraph{Computation}on the other hand produces that which was not entirely planned out by the
user, it creates unpredicted results. The input in this case is not the final product. This input constraints
the process towards a roughly imagined product; but does not completely define it. This is due to the
fact that computation is used in areas where humans are very limited; such as calculation and
processing of large numbers. The same applies to architectural modelling; one could use computation
to enter a formula that would render a cube, but that is never the case. Computation is utilised in
producing objects of high complexity, or in large numbers with a wide spectrum of variation that
would make it impossible for the user to manually model.

To reiterate the above stated; the product of computation --- even though the algorithms that govern
it and the input it needs are a product of the human mind --- is not solely a product of the human
mind, but a product of a parallel logic.

\section{Parallel Logic}

Algorithms solve problems in a finite number of steps; either deterministically, or
stochastically\footnote{Stochastic: Involving a random variable \cite{merriam03}}. The problems it
solves can be a specific problem or an exploration of the unknown. The logic it uses, was designed
by people, which might imply that algorithm is a subset of human logic. The fact is: it is not.

The way computers rationalise is different from how people think; it is in fact inconceivable by
humans, the same way a human's rationalisation is virtually impossible for a computer to emulate.
This is due to the fact that computers can process enormous amounts of abstract data and numbers,
such as the `Brute force' technique which processes `all' possibilities of a solution, which could be
thousands or millions of solutions. On the other hand, it cannot perceive `intuition', personal
preference; things that are unique to a human's mind and require emotion or subconscious
reactions.

\paragraph{Algorithms,}or procedures, are a description of steps to accomplish a specific task,
which ``\emph{allow abstraction and encapsulation of complexity and
re-usability}''\cite{hernandez06}. In abstraction; procedures have three parts:
\begin{enumerate}
  \item The name of the procedure, which is the handle by which the procedure is called.
  \item The arguments, which are the parameters used within the procedure. These are similar to
  ingredients in a recipe.
  \item The description of the procedure, which is the recipe itself. Within this process the user
  inputs values into parameters\footnote{Parameters are a series of arguments which take up values
  in a function, and are also the placeholder of variable value.}, the procedure makes calculations
  according to the description of the procedures and outputs the result to the user. The calculation
  part is usually hidden from the user, which is called encapsulation.
\end{enumerate}

\section{Geometric Modelling and Design as Algorithms}

The question is whether a paradigm of logic such as algorithms is capable of producing architectural
design. This can be answered by starting at the fact that architectural design is ultimately
composed of interrelated geometrical objects.

The way geometry is defined is very specific to the different types of geometry being defined. A
point is defined by three variables of space, a line is defined by two points each holding three
values in space\ldots etc, with the addition of more complex modelling techniques such as Boolean
operations. This nature of geometrical definitions requires specific procedures, each having a
different set of encapsulated parameters. Therefore, geometric modelling is a parametric
procedure. \cite{hernandez06}

\begin{figure}[htbp]
\centering
\begin{tikzpicture}[scale=4]
\draw [very thick] (0,0) --(0,2) --(2,2) --(2,0) --(0,0);
\draw (1,1) --(1,3) --(3,3) --(3,1) --(1,1);
\draw [->,dashed,thick] (0,0) --(1,1);
\draw [->,dashed,thick] (0,2) --(1,3);
\draw [->,dashed,thick] (2,2) --(3,3);
\draw [->,dashed,thick] (2,0) --(3,1);
\node [below] at (0,0) {\emph{(x=0,y=0,z=0)}};
\node [below] at (0,2) {\emph{(x=0,y=2,z=0)}};
\node [below] at (2,2) {\emph{(x=2,y=2,z=0)}};
\node [below] at (2,0) {\emph{(x=2,y=0,z=0)}};
\end{tikzpicture}
\vspace{5mm}
\caption[Procedures and Parameters in Algorithmic Modelling]{Procedures and Parameters in
Algorithmic Modelling. {\footnotesize Point procedure with three parameters, a square procedure with
points as parameters, and a cube extrusion procedure with the square as a parameter}}
\label{SqrAnalysis}
\end{figure}

In some cases, design can be also be defined by procedures, when it is possible to break down into a
step-by-step repetitive process, which also contain arguments (Parameters) that alter the design
with variable values.

When design is treated as a procedure; with parameters defining different designs with different
values; numbers, relations, shapes and operations are also treated as parameters. A cube for example
can be defined by extruding a square along it's perpendicular axe, which makes the square in this
case a parameter of the extrusion procedure (Figure \ref{SqrAnalysis}).

Algorithmic Design can be summarised as: ``\emph{A procedure carrying instructions in a systematic
order where all geometrical components that represent a design are
parameterised}''. \cite{hernandez06}
