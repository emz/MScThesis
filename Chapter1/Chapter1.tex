\chapter{Thermal Design and Building Envelope}
\pagenumbering{arabic}
\section{Introduction}

\subsection{The Science of Thermal Design}
Any good form of design, whether it be architectural or otherwise must have a sort of reasoning,
science or engineering as a base on which it is built to be useful, the same way good art must have
meaning to be tasteful. In our case we deal with architectural design, and a particular form of it;
thermal design.

The basis on which thermal design is built is mainly that of simple physics, and physiology. The
physics is the science that describes the building, it's components and it reactions to it's
external environment, sometimes also includes other elements of the equation of thermal design such
as clothing or furniture. As for the physiological part, it's concern is the occupant; the human
being. It describes the way our bodies feel and react to our environment, which is in this case the
environment created by the building as a reaction to it's own.

In this chapter we shall discuss the matter of thermal design and it's elements; the
factors of occupant, building and environment. An integral part of this
chapter also is dedicated to the definition of what is called the building 'envelope', it's thermal
behaviour and the projection of thermal design principles on this prominent part of buildings.

\subsection{Sustainability Aspects of Thermal Design}
An important question that may come to mind is how thermal design is concerned with green or
sustainable architecture. The answer is actually quite simple; in some countries such as the UK,
buildings have a 50\% share of the countries energy consumption, of which 60\% is used only by
houses, and half of which is used in space heating alone \cite{edwards96}. In other countries with
hotter climates such as Abu Dhabi, cooling energy consumption had risen from 1625 MW to 4750 MW of
power from 1990 to 2007 \cite{nauman07}. Obviously, heating and cooling of spaces is a major
element when it comes to energy consumption, which can be interpreted as fuel combustion as a means
of electricity generation in most cases, which is a main source of CO\textsubscript{2} emissions. So
if we were to state how thermal design is vital for sustainable building design, we could say that it allows
for less CO\textsubscript{2}, a better management of resources and a thermally comfortable space
for different activities.

\subsection{The Building Envelope}
\paragraph{What is a building envelope?}It is simply the building components which
separate the indoors from outdoors. These components include all `physical' barriers such as walls, roofs,
windows, doors and foundations \cite{HPO}. The main functions of a building envelope -or
`enclosure'- are \cite{straube05}:
\begin{enumerate}
  \item Structural support and load-bearing.
  \item Control of inward and outward flow of energy and matter.
  \item Aesthetic element of interior and exterior.
\end{enumerate}
Naturally, we are concerned only with the control of thermal energy, which will narrow our scope
throughout this chapter.
\section{Physics of Thermal Design}
As mentioned earlier in this chapter, we shall embark upon the study of basic physical and
physiological knowledge concerned with building thermal design, in this particular case we
will observe the information as listed and described mainly by Steven Szokolay \cite{szokolay08}.

\subsection{Physics of Heat}
\subsubsection{Heat and Temperature}
\paragraph{Heat,}is a form of energy while \textbf{Temperature} is it's symptom. Temperature has a
measurement scale of Celsius which is based on water's freezing and boiling points at 0 and 100
degrees, while the Kelvin is used for measuring temperature intervals. \textbf{Specific Heat}
provides the connection between heat and temperature. It is equal to the amount of energy; or heat,
required to raise the temperature of a unit mass of a substance by 1 degree, measured in J/Kg.K. This
attribute is indeed of utmost importance in terms of thermal design, as it defines how different
substances respond to heat. Some materials will reach specific temperatures much faster than others
and therefore in turn heat surrounding mediums faster. \textbf{Latent Heat} is the amount of heat
lost or absorbed when changing form one physical state to another.

\paragraph{Thermodynamics.}Thermodynamics deal with heat flow. The \textbf{First Law} of
thermodynamics states that energy cannot be created nor destroyed, but converted from one form to
another. The \textbf{Second Law} states that heat can flow in only one direction: from a high
temperature point to a low one. The magnitude of such a flow can be measured in two ways:
\begin{inparaenum}
  \item \emph{Heat flow rate}; the total flow in unit time through a defined area which is
  measured in Watt (W), or
  \item \emph{Heat flux density}; the rate of flow of heat through a unit area, measured in
  W/m\textsuperscript{2}.
\end{inparaenum}

\subsection{Heat Flow}
As stated in the second law of thermodynamics; heat flows from a high temperature point to a low
temperature point. This flow occurs in 3 distinct forms:
\begin{enumerate}\label{HeatExchange}
  \item \emph{Conduction}, which occurs within the same medium or through direct contact.
  \item \emph{Convection}, which occurs between different physical forms; such as gas and liquid, or
  through different mediums; such as liquid transporting heat from one body to another.
  \item \emph{Radiation}, which occurs from a body with warmer \emph{surface} to another which is
  cooler.
 
\end{enumerate}

The physics that governs these principles of heat transfer has much more detail to it; equations
that define the process and it's output and other factors that might affect the physical behaviour
of different materials regarding conduction, convection or radiation such as inevitable damage on
site that reduces their efficiency. Since we intend to handle matters of calculation regarding
thermal design by means of energy simulation applications, only the minimal amount of equations and
derivations will be mentioned later in the chapter when needed.

\subsubsection{Psychrometry}
This section will briefly describe the heat flow process when a new element is introduced to the
equation; \emph{humidity}. Our purpose is not to illustrate the means of calculation of different
processes of heat flow -as with the previous section- which is done through what is called the
\emph{Psychrometric Chart}. Our purpose is to demonstrate the general concepts of the process.
\paragraph{Air and Humidity.} Air is composed of a mixture of different gases, Oxygen ($O_2$) and
Nitrogen ($N_2$) being the prominent elements. Another element of great importance to our subject is
water ($H_2O$). It plays a key role in thermal comfort. Water is carried by the air in it's gaseous
state; water vapour. Air can carry a limited amount of water, this amount changes according to air
temperature, whenever this limit is reached, the air is said to be \emph{Saturated}. The amount of water vapour -or
\emph{Humidity} as we shall refer to it henceforward-, is measurable on two scales;
\emph{Absolute} and \emph{Relative} humidity. Absolute Humidity measures the weight of water vapour
in grams for each Kilogram of air (g/Kg), while relative humidity measures the percentage of maximum
amount of water vapour supported by air at a given temperature; or \emph{Saturation Humidity}.

\paragraph{Enthalpy.}It is the heat content of air, relative to the heat content 1 Kg of air at
$0^\circ$ Celsius and 0 humidity. We can derive two different components of air heat content; 
\begin{inparaenum}
\item \emph{Sensible Heat}; the amount of heat raising dry bulb temperature, and
\item \emph{Latent Heat}; the amount of heat evaporating water to it's gaseous state and forming
air humidity.
\end{inparaenum}

\paragraph{The Psychrometric Chart.}As mentioned earlier; the psychrometric chart
(figure \ref{PsychroChart}) is used to calculate different exchanges of heat and
humidity, these exchanges are called \emph{Psychrometric processes}. The processes can be summarised to:
\begin{enumerate}
  \item \emph{Heating} and \emph{Cooling}, at which the our point of reference on the chart moves
  right and left, respectively, are the processes of raising and lowering of dry-bulb temperature,
  respectively.
  \item \emph{Humidification}, is the evaporation of water by heat into the air reducing dry-bulb
  temperature and increasing humidity\footnote{The word \emph{Adiabatic Humidification} is sometimes
  used to designate that no sensible heat is introduced in the process.}. The point of reference
  moves diagonally up to the left on the chart as an indication of this process.
  \item \emph{Adiabatic Dehumidification}, is the removal of humidity by means of passing the air
  through a chemical sorbent\footnote{\emph{Sorb}: to take up and hold by either adsorption or
  absorption \cite{merriam03}}, which removes some of the moisture content, releasing heat and
  increasing dry-bulb temperature while reducing humidity.
\end{enumerate}

\begin{figure}[htbp] %[Here Top Bottom nextPage]
\centering
\includegraphics[width=17cm,angle=90]{./Images/1-PsychrometricChart}
\caption[The Psychrometiric Chart]{The Psychrometric Chart \cite{szokolay08} \label{PsychroChart}}
\end{figure}

\newpage
\section{Thermal Comfort}
Thermal design in principle is controlling indoor temperature, to increase or decrease gained heat.
If we were to say that our main concern is energy efficiency and sustainability, one might think
that this kind of endeavour does not concern the occupant; a simple matter of saving electricity
through energy efficiency. Such an assumption would be completely ignoring the fact that it is the
occupant's feeling of discomfort that triggers the need to heat or cool a space using air
conditioning; thus using more power.
\subsection{Equilibrium and The Factors of Thermal Comfort}
Thermal comfort is simply a matter of balance; the balance of gained and lost heat by the human
body such as metabolism and different heat exchange methods (section \ref{HeatExchange}). An
equation can be formulated to illustrate this exchange as:
\begin{equation}
M \pm Rd \pm Cv \pm Cd - Ev = \Delta S
\label{StoredHeat}
\end{equation}

\begin{minipage}{\dimexpr\textwidth-3cm}
{\footnotesize Where: $M=$ Metabolic rate, $Rd=$ net radiation exchange, $Cv=$
convection, $Cd=$ conduction, $Ev=$ evaporation and $\Delta S=$ change in stored heat.}
\end{minipage}\\\\\\
The factors of this equation of balance; or thermal comfort are described by S.
Szokolay \cite{szokolay08} as being of three main types or categories. These types are environmental,
personal and contributing factors.
\newline

\begin{table}[htbp]
\centering
\begin{tabular}{lll}
\hline
Environmental	& Personal			& Contributing Factors\\ \hline
Air Temperature	& Metabolic Rate	& Food and Drink\\
Air Movement	& Clothing			& Body Shape\\
Humidity		& State of Health	& Subcutaneous Fat\\
Radiation		& Acclimatisation	& Age and Gender\\
\hline
\end{tabular}
\caption[Factors of Thermal Comfort]{Factors of Thermal Comfort \cite{szokolay08}}
\label{FactorsOfComfort}
\end{table}

\subsection{Homoeothermy}
\paragraph{Homoeothermy,}is the ability of a body to adjust it's internal temperature through
different means such as sweating to reduce body heat by evaporation, or shivering to induce heat.
Basic methods though are not that prominent, which are \emph{vasoconstriction} and
\emph{vasodilation}; the reduction and increase of blood flow to the skin, respectively, to conserve
body heat in cold climates and to release it in warmer climates. These are important factors to be
considered, so as to not treat the body as a passive thermostat, and compensate for such reactions
when designing a thermal space. There are other physiological changes in reaction to cold and warm
weather that occur on the long term such as change in metabolism rate that we will not delve into.

\paragraph{Psychological perception,}is another aspect of how we conceive temperature. This is
mainly a matter of expectation. Any given temperature will be perceived differently according to
prevailing temperatures of a particular month or season, so if we were to say that outdoor
temperature and wind speed are $24^\circ C$ and $0.2 m/sec$, it might be perceived as mildly cold in
a summer season, or mildly warm in an autumn season. The following figure
(\ref{PsychoPhysioModel}) illustrates the psycho-physiological model of thermal perception
\cite{szokolay08}.

\begin{figure}[htbp]
\includegraphics[width=13.5cm]{./Images/2-PsychoPhysioModel}
\caption[Psycho-Physiological Thermal Perception Model]{Psycho-Physiological Thermal Perception
Model \cite{szokolay08} \label{PsychoPhysioModel}}
\end{figure}

\subsection{Comfort Indices}
Since thermal comfort is affected by multiple environmental aspects, a unified index was needed to
define environmentally accepted conditions. Many indices have proposed and accepted over the years
including the famous chart made by Victor Olgay. One index which has been introduced lately and is
widely accepted is the ET* (ET star), or it's standardised version the \textbf{SET}. The SET index
is superimposed upon the Psychrometric chart (Fig. \ref{PsychroChart}) coinciding with DBT at 50\%
RH curve until it reaches $14^\circ C$, then slanting gradually, demonstrating that higher humidity
lowers the tolerance for temperature of occupants.

\section{Climate}
This section deals with the element of thermal design which defines the external environmental
factor; climate. Although it is a complex system with many aspects (radiation, wind speed, wind
temperature, precipitation\ldots etc.), we shall discuss exclusively the environmental aspect of the
Sun. This choice has been made in order to simplify experimentation with
algorithmic design of building envelope through our focus on one element of the heat exchange process.

\subsection{The Sun}
Szokolay \cite{szokolay08} argues that the aspects essential for a designer to understand are the
movement of the Sun\footnote{This is a reference to it's apparent movement as a result of the actual
movement of Earth in it's orbit.}, and how to utilise or diminish it's effect.\\
The Earth's orbit around the sun has unique characteristics such as it's slanted rotational axis
relative to it's orbital axis around the sun, creating difference in temperatures across the year;
creating seasons. This phenomenon also results in the difference in day and night lengths depending
on the distance from the Earth's equinox\footnote{The point at which the Earth's equator is on the
same plane with the Sun's centre.}. These characteristics are discussed in detail in many
publication, but our main concern is the resulting movement of the sun.

\subsubsection{Sun-path Diagrams}
The essential tool of studying the Sun's movement is the Sun-path diagram. This diagram displays the
Sun's position at any given date, time and global position. The position of the Sun is marked all
year round on projection of the sky represented as a dome or hemisphere, with the the desired
position on land being it's centre, this view is called a \emph{Lococentric} view (Figure
\ref{Lococentric}) The most common projection is called the \emph{Stereographic} projection (Figure
\ref{Stereographic}).

\begin{figure}[htbp]
\centering
\includegraphics[width=10cm]{./Images/3-Lococentric}
\caption[Lococentric Sun Path View]{Lococentric View \cite{szokolay08}}
\label{Lococentric}
\end{figure}

\begin{figure}[htbp]
\includegraphics[width=\textwidth]{./Images/4-Stereographic}
\caption[Stereographic Sun-Path Projection] {Stereographic Sun-Path Projection \cite{szokolay08}}
\label{Stereographic}
\end{figure}

\subsubsection{Solar Radiation}
Perceivable solar radiation by humans is partitioned into three ranges:
\begin{enumerate}
  \item UV radiation, which is of $20-380nm$ wave-length.
  \item Visible light, ranging in a spectrum of violet to red, $380-700nm$ length.
  \item Short infra-red\footnote{Also described as \emph{thermal radiation}}, ranging from
  $700-2300nm$.
\end{enumerate}
The amount of irradiation\footnote{Energy flow density integrated over a period of time, measured in
$Wh/m^2$} differs from one location to the other. There are causes of this
phenomenon\cite{szokolay08}, which are:
\begin{enumerate}
  \item Angle of incidence, changing according to the angle of the earth's axis relative to it's
  solar orbit.
  \item Atmospheric depletion, depending on altitude; the distance which the radiation has to
  travel.
  \item Duration of sunshine, depending on day length and local topography.
\end{enumerate}

\section{Thermal Behaviour of Buildings}
Buildings have a thermal system very similar in concept to thermal comfort
(equation \ref{StoredHeat}), it is an equation of heat gain and loss, but with the minor
difference that thermal comfort is essentially defined by equilibrium, while thermal behaviour of buildings is
not.\\ The defining equation of thermal behaviour of buildings is as follows:
\begin{equation}
Qi \pm Qc + Qs \pm Qv - Qe = \Delta S
\end{equation}

\begin{minipage}{\dimexpr\textwidth-3cm}
{\footnotesize Where: $Qi=$ internal heat gain, $Qc=$ conduction heat gain or loss, $Qs=$ 
solar heat gain, $Qv=$ ventilation heat gain or loss, $Qe=$ evaporation heat loss and $\Delta S=$
change in stored heat.} \end{minipage}\\\\\
An important fact is that solar radiation is the most significant energy input in the
equation\cite{szokolay08}, this is reassuring since we have chosen to take the solar element into
account exclusively throughout the entire dissertation as mentioned earlier.

\subsection{Solar Control}
The idea behind algorithmic design -in our particular case- is controlling the design process
in order to produce a favourable condition. Since we have narrowed our scope to the issue of solar
design, our favourable condition will depend on the introduction or prevention of solar radiation.
The unfavourable solar radiation period causing overheating is marked on the sun path diagram, showing
the angles at which the sun casts it's unwanted rays, then superimposed on a diagram with the
opening's orientation on which our shading instrument is designed creating a shading mask.

\subsubsection{Shading Design}
\label{Shading}
Shading is the proven most efficient way of controlling solar radiation externally. External shading
is divided into to categories:
\begin{enumerate}
  \item	\textbf{Vertical Shading}, this type is characterised by by horizontal shadow angles (HSA),
  examples of which are vertical louvres (fig. \ref{HSA}).
\begin{figure}[htbp]
\centering
\includegraphics[width=5cm]{./Images/5-HSA}
\caption[Horizontal Shadow Angle]{Horizontal Shadow Angle \cite{szokolay08}}
\label{HSA}
\end{figure}
  \item \textbf{Horizontal Shading}, are characterised by vertical shadow angles (VSA), examples of
  which are horizontal louvres and canopies (fig. \ref{VSA}).
\begin{figure}[htbp]
\centering
\includegraphics[width=3cm]{./Images/6-VSA}
\caption[Vertical Shadow Angle]{Vertical Shadow Angle \cite{szokolay08}}
\label{VSA}
\end{figure}
  \item \textbf{Solar VSA}, identical to VSA unless not on the vertical plane of the surface normal
  of the building (fig. \ref{EggCrate}).
  \item \textbf{Egg-crate Shading}, complex shading devices combining all previous shading masks.
\begin{figure}[htbp]
\centering
\includegraphics[width=6cm]{./Images/7-Egg-Crate}
\caption[Egg-Crate Shading Devices]{Egg-Crate Shading Devices \cite{szokolay08}}
\label{EggCrate}
\end{figure}
\end{enumerate}

\subsubsection{Solar Heat Gain}
Solar heat gain is a product solar radiation, which is divided into 
\begin{inparaenum}
	\item direct beams, 
	\item diffused beams and 
	\item occasional reflected beams.
\end{inparaenum}
Solar heat gain is calculated differently depending on whether it is opaque or transparent.

\paragraph{Transparent elements,}such as glazing, receive solar radiation and produce and
amount of solar heat gain depending on it's surface area ($A$) and the materials solar gain
factor ($\theta$). The incident rays are divided into transmitted rays($\tau$), reflected rays
($\rho$) and absorbed rays ($\alpha$).
\begin{equation}
\tau + \rho + \alpha = 1
\end{equation}
The absorbed portion of solar rays is then emitted to the outside and the inside through convection
and radiation, adding to the amount of heat transmitted through the glass to the inner space, thus
producing solar heat gain.
\begin{equation}
Qs=A\times G\times \theta \label{TransHeatGain}
\end{equation}

\paragraph{Opaque elements}are treated differently. The input radiant heat depends on the surface
absorptance ($\alpha$): 
\begin{equation}
Q_{in}=G\times A\times \alpha
\label{OpInRad}
\end{equation}
The heat input will raise surface temperature ($T_s$), causing heat loss to the air, which depends
on surface conductance ($h$): $Q_{loss}=A\times h\times (T_s\times T_o)$, $T_o$ being air
temperature.\\
When the surface reaches equilibrium; i.e. $Q_{in}=Q_{loss}$, or:
\begin{equation}
G\times A\times \alpha = A\times h\times (T_s\times T_o)
\end{equation}
We can derive $T_s$ as:
\begin{equation}
T_s=T_o+G\times \frac{\alpha}{h}
\end{equation}
Bear in mind that this value is of a ``sol-air'' temperature; which neglects the transfer of
heat from the surface into the body of the element \cite{szokolay08}. Thus, we deduce that the heat
flowing through an opaque element can be calculated as:
\begin{equation}
Q_c=A\times U \times (T_s-T_i)
\newline
\end{equation}
\center{ \footnotesize Where $U$ is the air to air thermal transmittance, \\and $T_i$ is the internal
air temperature}
\flushleft

\subsection{Steady State Heat Flow}
\paragraph{Conduction heat flow,}(see section \ref{HeatExchange}) is calculated as the product of
$A\times U\times \Delta T$. In our case we will use the accumulated area and transmittance of the
building envelope; referred to as \emph{envelope conductance}.
\begin{equation}
Q_c=\sum(A\times U)\times \Delta T
\label{EnvConduct}
\end{equation}
{\footnotesize Where $\Delta T$ is equal to the difference between outdoor and indoor
temperatures $T_o-T_i$}

\paragraph{Thermal bridges,}
All heat flow calculations hitherto are true only when heat flow is ``one-dimensional''
\cite{szokolay08}; assuming that incident radiation is in the surface normal direction, and that the
envelope element is quite large, flat, solid and of uniform construction material and geometry---which is rarely the case, if
ever.\\
This shows that heat flow calculations are an approximation. Here the concept of thermal bridges
should be introduced, which cause heat to flow in two or three dimensions, caused by geometrical
variation or construction material differences in one building element (figure
\ref{ThermalBridges}).
\begin{figure}[htbp]
\centering
\includegraphics[width=6cm]{./Images/8-ThermalBridges}
\caption[Thermal Bridges]{Thermal bridges caused by geometrical and material variation
\cite{szokolay08}} \label{ThermalBridges}
\end{figure}

\section{Thermal Design Variables}
An integral part, and arguably the most important one of this chapter, is where we demonstrate the
different design variables affecting the building thermal performance, or it's envelope. This
importance comes from the fact that these variables will be the basis on which we will choose our
form generation parameters later on.\\
The most influential traits of any given building as argued by Szokolay \cite{szokolay08} are shape,
fabric, fenestration and ventilation. We will demonstrate different attributes with the exception of
ventilation.\\ \vspace{0.5cm}
\begin{enumerate}
  \item \textbf{Shape}
  	\begin{enumerate}
    	\item \emph{Surface-to-volume ratio}\\ As illustrated before (eqn. \ref{EnvConduct}), heat gain
    	is partially a product of surface area, therefore it is essential to decrease or increase the
    	surface area of a building in relation with it's volume, or to be practical; it's footprint.
   		For example; if we were to decrease the surface area, the optimum shape would be a
    	spherical/hemispherical one. 
    	\item \emph{Orientation}\\ In order to utilise heat gain and heat dissipation properly, one
    	must consider building shape ration and orientation. Usually elongated northern and southern
    	fa\c{c}ades with a ratio of 2 to 1.3 are suitable for most cases \cite{szokolay08}.
  	\end{enumerate}
  \item \textbf{Fabric}\footnote{Resistive, reflective and capacitive insulation have been
  omitted from this list, due to it's irrelevance to our scope}
  	\begin{enumerate}
    	\item \emph{Shading}\\ \emph{See section \ref{Shading}}
    	\item \emph{Surface material properties}\\ This is a matter of absorptance and reflectance
    	(refer eqn.\ref{OpInRad}). Generally speaking; a bright colour is reflective, while a dark
    	one absorbs more energy; or heat.
  \end{enumerate}
  \item \textbf{Fenestration}
  	\begin{enumerate}
  	  \item \emph{Size, position and orientation}\\ A fundamental aspect of fenestration and
  	  radiation input control\footnote{Also an essential variable of ventilation control}.
  	  \item \emph{Glazing material}\\ Single glazing, double glazing\ldots in addition to glass
  	  quality and different unique traits the control heat input.
  	  \item \emph{Closing mechanism}\\ Solutions include fixed glass, louvres and opening sashes.
  	  \item \emph{External Shading}\\ In relation with windows or envelope voids.
  	\end{enumerate}
\end{enumerate}
