\chapter{Generative and Performative Algorithms}

\section{Introduction}

Today, two main paradigms of algorithmic design are clearly dominant; generative design and performative design \cite{fasoulaki08}. Generative design is the essential \emph{rule-based\footnote{Consisting of a solid set of consecutive instructions that govern an action to produce a desired outcome}} algorithmic form of algorithmic architecture. It is governed by repetitive patterns, equations and shape transformations.  Performative design on the other hand, is optimisational; modifying and reshaping to adapt to a certain simulated environment.

\section{Generative Algorithm}

Generative architecture can be defined as a design process involving generative systems in the form of a computer program that can be fully automated or step-by-step controlled. It would have an algorithm (rule) governing the transformations of initial shapes through variables or parameters; `generating' results, and selecting the best variant according to the given criteria. \cite{arida04}

The tools and key components of generative design are the systems which define the method of form exploration. They can be defined as scripts or sets of geometrical transformations \cite{arida04}. These systems ---as mentioned earlier--- are rule based, essentially formal or visual and are often based on systems borrowed from biology and mathematics. Some popular systems widely used today include Cellular Automata, L-Systems, Voronoi Diagrams, Fractals and Shape Grammars.

\subsection{Cellular Automata}

``Cellular Automata is a computational method which can simulate the process of growth by describing a complex system by simple individuals following simple rules. The concept of simulating growth was introduced by John von Neumann (1951) and further developed by Ulam (1962) in the are of simulating multi-state machines. The concept gained great popularity when Martin Gardner (1970) John Conway's ``Life'', a game that generated two dimensional patterns''\cite{krawczyk02}. Later on, Stephen Wolfram would begin research on the subject in 1984, to write a book on the subject called \emph{A New Kind of Science} \cite{wolfram02}. The book would discuss the patterns created by cellular automata as simple rules capable of generating extremely complex patterns that mimic patterns of growth occurring in nature, among other things; describing his findings as a new paradigm of scientific research touching on numerous fields of science.

\begin{figure}[htbp]
\centering
\includegraphics[width=\textwidth]{./Images/1-CellAutoEg}
\caption[Cellular Automaton]{An example of cellular automata showing (a) the sequence of generations and (b) the governing rule.\cite{wolfram02}}
\label{CAEg}
\end{figure}

The concept of cellular automata is that an initial configuration of at least one cell occupying an empty space would evolve and procreate into a pattern using a rule that decides birth, life and death cells through the preceding related cells (Fig. \ref{CAEg}). The pattern shown is a two dimensional growth pattern used by Stephen Wolfram. Other forms of cellular automata generation include 3-dimensional ones, which are the architects main interest.

The method of generation in 3-dimensional cellular automata remains basically the same as with 2-dimensional ones; with an initial configuration of blocks occupying spaces and a rule that governs the shape of each generation (Fig \ref{3dCA}).

\begin{figure}[htbp]
\centering
\includegraphics[width=0.7\textwidth]{./Images/2-3dCA}
\caption[3-Dimensional Cellular Automata]{3-Dimensional cellular automata and terminology.\cite{krawczyk02}}
\label{3dCA}
\end{figure}

The utilisation of basic rules of cellular automata however does not suffice to produce architecturally sound models with appropriate internal spaces in most cases (Fig \ref{3dCAArch}). In order to produce such forms, a degree of optimisation and modification is required. A fairly simple but effective solution is manipulation of cell interpretation. Examples of such manipulation are stretching cells to overlap forming adequate internal spaces, merging the overlapping cells and interpreting the envelope of merged cells as splines (Fig \ref{CA Optimisation}).

\begin{figure}[htbp]
\centering
\includegraphics[width=0.7\textwidth]{./Images/3-3dCAArch}
\caption[Cellular Automata Architectural Inadequacy]{An example of CA rules inadequacy for architectural forms without modification \cite{krawczyk02}}
\label{3dCAArch}
\end{figure}


\begin{figure}[htbp]
\centering
\includegraphics[width=0.7\textwidth]{./Images/4-3dCAArchOpt}
\caption[Cellular Automata Architectural Optimisation]{Optimisation of cell interpretation for architectural form \cite{krawczyk02}}
\label{CA Optimisation}
\end{figure}

Multiple interpretations of the manipulated generations (Fig \ref{CA ArchForm}) were created with structural support. The architectural forms shown appear more developed, nevertheless; it was subject to several modifications of the original cellular automata model which required user intervention at early stages of the process, which would prove CA to be a rather raw method of architectural form generation.

\begin{figure}[H]
\centering
\includegraphics[width=\textwidth]{./Images/5-3dCASpaces}
\caption[Architectural Interpretation of CA Generations]{Architectural Interpretation of CA Generations \cite{krawczyk02}}
\label{CA ArchForm}
\end{figure}

\subsection{Voronoi Diagrams}

``Considered as early as 1644 by Ren\'{e} Descartes but are named after the Russian Mathematician Georgy Fedoseevich Voronoi who studied and defined the general n-dimensional\footnote{\emph{Euclidean Space}: The definition of all points in space by three coordinates; namely $x,y \text{ and } z$} case in 1907''\cite{fasoulaki08}. A Voronoi Diagram is a way of decomposing a space into regions\footnote{Voronoi Diagrams are a class of patterns called Dirichlet tessellations}, this process initiate with a set of points in space; called generating points, the partitioning lines are then drawn from an equal distance from the aforementioned points creating polygons or cells (Fig. \ref{VDiag}).

\begin{figure}[htbp]
\centering
\includegraphics[width=0.8\textwidth]{./Images/6-VoronoiDiagram}
\caption[Voronoi Diagram]{Voronoi Diagram \cite{fujita00}}
\label{VDiag}
\end{figure}

Voronoi Diagrams are naturally occurring patterns such as crystalline formations, bee honeycombs and animal coat patterns among other things, making it a suitable model of form generation and optimisation as well in terms of structure \cite{friedrich08} and environmental performance\label{NaturalOpt}. The main form of utilisation however is that of internal space division or partitioning (Fig. \ref{VDInternal}) and urban design and planning (Fig. \ref{VDUrban})

Although the common form of utilisation of Voronoi Diagram might denote a 2-dimensional constraint, it is also used in 3-dimensional space generating surfaces points in 3D space.

\begin{figure}[htbp]
\centering
\includegraphics[width=\textwidth]{./Images/7-UrbanVoronoi}
\caption[Voronoi Diagram and Urban Design]{An example of Voronoi Diagram utilisation in urban design (with manual optimisation of final results) \cite{friedrich08}}
\label{VDUrban}
\end{figure}

\begin{figure}[htbp]
\centering
\includegraphics[width=\textwidth]{./Images/8-StructureVoronoi}
\caption[Voronoi Diagram Space Decomposition]{An example of Voronoi Diagram internal space decomposition \cite{friedrich08}}
\label{VDInternal}
\end{figure}

\subsection{Fractals}

"Fractal Geometry is the study of mathematical shapes that display a cascade of never-ending, self-similar, meandering\footnote{\emph{Meander}: to follow a winding or intricate course \cite{merriam03}} detail as one observes them more closely\ldots Fractal geometry is a rare example of technology that can reach into the core of design composition" \cite{bovill96}.

Originated by Benoit B. Mandelbrot in 1975 \cite{fasoulaki08}, the term \emph{fractals} defines a mathematical rule to produce geometrical shapes that are self-similar\footnote{Reduced copy of the whole}. A fractal is produced by defining an initiator shape and a rule that replaces this shape with a smaller set of copies of the original shape, such as a symmetrical combination of two copies of the original shape. The process goes on in recursion and repetition with virtually no limit to the number of iterations. Examples of this process are: the Koch curve (Fig. \ref{KochCurve}), Sierpinski Gasket, Contor Sets and Julia Sets (Fig. \ref{Fractal} \subref{JuliaSets}).

\begin{figure}[htbp]
\centering
\includegraphics[width=0.8\textwidth]{./Images/9-FractalGeom}
\caption[The Koch Curve]{An example of fractal geometry: The Koch Curve \cite{falconer03}}
\label{KochCurve}
\end{figure}

Fractals are a naturally occurring phenomenon, such formations of snow flakes, clouds, fern leaves (Fig \ref{Fractal}\subref{Fern}) and coastlines. As mentioned before (\emph{see page} \pageref{NaturalOpt}); the derivation of generative systems from natural formations can be beneficial in optimisational terms of building performance.

\begin{figure}[htbp]
\centering
	\subfloat[Fern Fractal]{{\includegraphics{Images/10-Fern}\label{Fern}}}
	\hspace{2cm}
	\subfloat[Julia Sets]{{\includegraphics{Images/11-JuliaSets}\label{JuliaSets}}}
	\caption[Fractal Forms]{Fractal Forms \cite{falconer03}}
	\label{Fractal}
\end{figure}

\subsection{Shape Grammars}

A project by George Stiny and James Gips in 1977; shape grammars were intended to be a scientific approach to the language and composition of design. The main goal of shape grammars was to make the design process detached from individuals' minds and externalised to something that can be transmitted and modified. A distinctive feature of shape grammars is the visual method of form generation rather than the symbolic or numerical.

As with cellular automata and fractals; the process of generation with shape grammars starts with an initial shape, and is transformed through a given rule. Rules in shape grammars take the form of: \begin{equation} A \rightarrow B \end{equation}
Whenever a shape matching the left side of the rule $A$ occurs, the rule is applied (Fig. \ref{ShapeGrammarsRule}). As described in a previous section (\ref{AlgoGeoModel}), the type of transformation applied to the shape can addition, subtraction, rotation\ldots etc. 

\begin{figure}[htbp]
\centering
\includegraphics[width=\textwidth]{./Images/12-ShapeGrammarsRule}
\caption[Shape Grammars Rule]{A Shape Grammars Rule: \emph{(a) rule, (b) transformation} \cite{arida04}}
\label{ShapeGrammarsRule}
\end{figure}

Shape grammars also have the ability to apply the rules to emergent shapes; not predefined in the grammar. Shape grammars can be used as a synthesizer of shapes be generation, or as an analysis tool of complex shapes; decomposing them into simple ones. Among the many real world applications of shape grammars, a relatively new one was described by D. Shelden \cite{shelden02}; which was during the preparation of contract documents of the \emph{Experience Music} project. The case was the subdivision of a cladded surface for fabrication of the cladding sheets. The surface was divided using a shape grammar rule, but upon receiving the requirements of the fabricator which were smaller sheets; a different rule was used to subdivide the surface (Fig. \ref{SubdivisionSG} -- \ref{SubdFab}).

\begin{figure}[htbp]
\centering
\includegraphics[width=0.5\textwidth]{./Images/13-SubdivisionRule}
\caption[Shape Grammar Subdivision]{The subdivision of a surface by shape grammars for fabrication of the sheets \cite{shelden02}}
\label{SubdivisionSG}
\end{figure}

\begin{figure}[htbp]
\centering
\includegraphics[width=\textwidth]{./Images/14-SurfaceFabrication}
\caption[Fabrication Surface Subdivision]{The subdivision of building envelope for fabrication \cite{shelden02}}
\label{SubdFab}
\end{figure}

\newpage
\subsection{L-Systems}

L-System; an abbreviation of Lindenmayer-systems after Aristid Lindenmayer, was created in 1968 as a mathematical simulation of plant growth \cite{fasoulaki08}, consisting of four elements: \begin{inparaenum} \item a starting configuration string, \item a set of rules, \item constraints and \item variables.\end{inparaenum}

L-Systems are very similar to Shape Grammars, with the exception of being represented textually not spatially \cite{arida04}, but operate in a recursive manner such as that of Shape Grammars. The form by which rules of context-sensitive L-Systems is expressed is: \begin{equation} B<A>C\rightarrow X \end{equation}\ldots where $A$ would produce $X$ if and only if $A$ is surrounded by $B$ on the left and $C$ on the right. The rule form of context-independent L-Systems is: \begin{equation} A \rightarrow B \end{equation}

\begin{figure}[htbp]
\centering
\includegraphics[width=\textwidth]{./Images/15-L-SystemLeaf}
\caption[L-System Compound Leaf]{Development of a compound leaf shape \cite{csiro96}}
\label{LSysLeaf}
\end{figure}

\newpage
\section{Performative Algorithms}

Having observed some the prominent methods of generative design, one could define performative design by saying that unlike generative methods performative design deals with how the building performs; not how the building looks. In order to measure a specific aspect of any given building's performance; one has to define the variables, constraints and performance criteria. But although the aim of such practice would be to measure how the building performs, the feedback would certainly change how the building looks.

When speaking of green and sustainable design, performative design always comes to mind. However; it does not deal with the environmental impact of buildings, which rules out the capability of producing green design. Nevertheless; in recent years it has become the quintessential method of production of sustainable design. This owes to the fact that performative design, through simulation and selection can produce designs that are highly efficient in terms of energy consumption and waste production.

Performative design differs from generative design in that it does not rely solely on an algorithm for generation, but uses optimisation and simulation as tools. These tools are the main subject of discussion in this section.

\paragraph{Optimisation}is the selection of the \emph{optimum} solution. Optimisation is a widely used system in engineering, considered as a problem-solver; optimisational algorithms are called search methods. As mentioned in a previous section (\ref{ParallelLogic}), computers have the power to delve into what humans cannot. In this case; it is the power to solve problems with high complexity requiring enormous amounts of computation.

\newpage
\subsection{Genetic Algorithm}

Genetic algorithm is a heuristic\footnote{Of or relating to exploratory problem-solving techniques that utilise self-educating techniques to improve performance\cite{merriam03}} search method for solving optimisation problems simulating biological evolution \cite{fasoulaki08}. Genetic Algorithm has unique terminology due to it's origins in biological evolution science.

``Under GA terminology, a solution to a problem is an \emph{individual}, and the group solutions existent at each stage is a \emph{population}. Each time a new population of individuals is created is called a \emph{generation}. In binary GAs\ldots each individual is represented by a binary string called a \emph{chromosome}, which encodes all parameters of interest corresponding to that individual. A chromosome is formed of \emph{alleles}; the binary coding bits. The fitness of any particular individual corresponds to the value of the objective function at that point''. \cite{caldas01}

\begin{figure}[htbp]
\centering
\begin{tikzpicture}[node distance=1.5cm, auto]
\node [rectangle, draw, rounded corners, minimum height=1cm] (init) {Generating Initial Population};
\node [rectangle, draw, rounded corners, minimum height=1cm, below of=init] (scaling) {Scaling};
\node [rectangle, draw, rounded corners, minimum height=1cm, below of=scaling] (select) {Selection};
\node [rectangle, draw, rounded corners, minimum height=1cm, below of=select] (cross) {Crossover};
\node [rectangle, draw, rounded corners, minimum height=1cm, below of=cross] (mutation) {Mutation};
\node [rectangle, draw, rounded corners, minimum height=1cm, below of=mutation] (elit) {Elitist Model};
\node [shape=diamond, draw, shape aspect=2, node distance=2.5cm, below of=elit] (term) {Termination Condition};
\node [shape=circle, draw, shape aspect=2, node distance=2.5cm, below of=term] (end) {End};

\draw [-latex', thick] (init) -- (scaling);
\draw [-latex', thick] (scaling) -- (select);
\draw [-latex', thick] (select) -- (cross);
\draw [-latex', thick] (cross) -- (mutation);
\draw [-latex', thick] (mutation) -- (elit);
\draw [-latex', thick] (elit) -- (term);
\draw [thick] (term) -- (-4,-10);
\draw [thick] (-4,-10) -- (-4,-1.5);
\draw [-latex', thick] (-4,-1.5) -- (scaling);
\draw [-latex', thick] (term) -- (end);
\node [above] at (-3.5,-10) {No};
\node [left] at (0,-11.6) {Yes};
\end{tikzpicture}
\caption[Genetic Algorithm Flowchart]{The sequence and loop of GA's}
\label{GAFlw}
\end{figure}

The process of problem solving in GA is fairly simple (Fig \ref{GAFlw}), despite it's ability to solve highly complex problems. Initially; a population is randomly generated with a sufficient amount of chromosomes allowing for a wealth of genetic information to be processed and therefore statistically provides a better solution, although the bigger the population; the more computing power is needed. 

Individuals are then selected on a basis of fitness defined by the user prior to the process. At this stage, a number of genetic operators become in effect, such as \emph{crossovers}. Crossovers are swapping of parts of two chromosomes selected for their fitness, producing a new offspring of individuals containing some genetic material from each \emph{parent} individual. This process is repeated and reiterated to produce more refined generations gradually through selection.

Another function of GA is \emph{mutation}, which occurs randomly at a predefined to affect some chromosomes in the intermediate phase between any two crossovers. The idea of mutation is to introduce alterations to the selection of chromosomes, and therefore allowing for genetic diversity at later stages of the selection process, assisting the search by escaping the local optima. Nevertheless, it's frequency is adjusted as to allow for genetic diversity without producing a generation of crossovers that is completely different from it's predecessor.

At the final stage of the search process, an elitist model is created and presented as the `best' set of solutions, which meet the termination criteria; ending the search. The difference between GA and other search is method, is that GA is a probabilistic search method, not a deterministic one, which searches through a population in different points in parallel not a single point. GA is considered a last resort when it comes to search methods, when all mathematical methods have failed due to the complexity and size of the search population, producing the best solution of the generated population, but not the optimum solution.
