\documentclass[a4paper,twoside,12pt,openright,final,oldfontcommands]{memoir} %use oldfontcommands
%to enable arabtex and utf8


%\usepackage[bindingoffset=5mm]{geometry} %try memoir's built in commands

\usepackage[T1]{fontenc}

\usepackage{color}
\usepackage{algorithm}

\usepackage{caption}
\captionsetup{font=small,labelfont=bf}

\usepackage{arabtex} %Always load after caption package
\usepackage{utf8}
%to use arabic use the following
%\begin{arabtext}
%\setcode{utf8}
%<insert arabic text here>
%\end{arabtext}

%\usepackage{fancyhdr}	%not quite useful when you use memoir

\usepackage{graphicx}
\usepackage{float} %enables \figure[H] or [h!]

\setcounter{secnumdepth}{3} %enable numbering of subsubsections (level 3)

\usepackage{fixltx2e} %adds \textsubscript functionality
\usepackage{paralist} %adds inparaenum

\usepackage{hyperref}
\hypersetup{pdftex,linktocpage=true,bookmarks=true,colorlinks=true,citecolor=blue,filecolor=black,linkcolor=blue,
urlcolor=black,pdfauthor={Ayman Elmasry}, pdftitle={Algorithmic Form Generation: A Thermal Building
Envelope Design Approach}}
\usepackage[all]{hypcap}

%\renewcommand{\bibname}{References}

\title{\textsc{Algorithmic Form Generation:\\
\large A Thermal Building Envelope Design Approach}}
\author{}
\date{}

\begin{document}
\frontmatter

\pagestyle{plain}

\includegraphics[width=2cm]{./images/AinShams}\\
\footnotesize
Ain Shams University\\
Faculty of Engineering\\
Department of Architecture\\

\begin{center}
	\Large

\textbf{Algorithmic Form Generation:\\
	A Thermal Building Envelope Design Approach
}\\

\vspace{0.5cm}
\small
By:\\
\normalsize
\textbf{Ayman Osama Elmasry}\\
BSc Architectural Engineering\\
Higher Technological Institute\\
\vspace{0.5cm}
A Thesis Submitted to the Faculty of Engineering, Ain Shams University, in Partial Fulfilment of the Requirements for the Degree of Master of Science in Architectural Engineering\\
\vspace{0.5cm}
Under the Supervision of:\\
\vspace{0.5cm}
\textbf{Asst. Prof. Dr. Ahmed Atef}\\
Assistant Professor of Architectural Engineering, Faculty of Engineering, Ain Shams University\\
\vspace{0.3cm}
\textbf{Dr. Hazem Eldaly}\\
Lecturer of Architectural Engineering, Faculty of Engineering, Ain Shams University

\end{center}

\cleardoublepage
\pagestyle{ruledsmallhd}
\setlength{\parskip}{0.7cm}

\begin{abstract}
	The process of thermal design of building envelopes is studied in this thesis from the standpoint of computation --- the use of computers as tools to assist the architect in find the optimum solution in terms of thermal performance. The basic concepts of Algorithms are explained, in pure terms of programming, and then projected onto the geometrical nature of architectural modeling, to highlight what is called \emph{Parametric Modeling}.\\
	
	Prominent examples of algorithms are presented and categorised under two main paradigms, which are the \emph{Generative} approach, and the \emph{Performative} approach, each having their distinct methods of handling the design process. The algorithm examples are briefly explained and some of their possible applications in the field of architectural design are presented. Special emphasis is put on performative algorithms and the underlying mathematical process of optimisation, which would be integral to the process later on.\\

	The main research problem of thermal design is then presented. Elementary scientific concepts that govern the thermal reactions of both the building and the occupants is briefly explained. The most important element discussed thereof is the determination of the \emph{Design Variables} which affect the process of parameterisation of specific elements of the building envelope for optimisation by the algorithm.\\

	An analytical and comparative study is made of previous research and experiments that utilise generative and performative algorithms, and in the process differentiating between the driving forces of design in both cases, and the basic procedure for implementation.\\

A general methodology is formulated out of the studied examples and previously explained concepts, to be applied to the process of thermal design of building envelopes. The ideas and scientific principles presented in the thesis are discussed in detail from the standpoint of algorithmic optimisation of building envelopes thermal performance. Finally, the sequence of work is ultimately divided into four consecutive stages --- namely \begin{inparaenum} \item Analysis of Envelope Form, \item Thermal Criteria and Target Performance, \item Virtual Modeling and Simulation, and finally \item Programming and Optimisation.\end{inparaenum} Afterwards, a number of theoretical examples are presented as showcase examples of implementation of the proposed methodology.
\end{abstract}


\maxtocdepth{subsection}
\cleardoublepage
\tableofcontents

\cleardoublepage
\listoffigures

\cleardoublepage
\listoftables

\cleardoublepage

\chapter{Introduction}

With the noticeable increase of interest in green architecture and building technologies in the last two decades, architectural designer are taking a more interdisciplinary approach to the conceptual design phase of buildings to consider various systems and services at very early stages of the design process.

This has been greatly beneficial as the early integration of different disciplines allows consideration of building systems and services that highly impact the performance of the building; with energy consumption being the most important element and indicator of building performance.

According to numerous studies in different scientific disciplines that concern housing, building and construction; the heating or cooling of internal building spaces --- residential buildings in particular --- are the most consuming of all building functions, reaching up to 40\% of total annual energy consumption of buildings.

The sustainable approach to thermal design of buildings dictates that passive design and control mechanisms be considered first before any mechanical or electrical system is considered. In turn, the usual approach of the architect is to do survey, study and analyse the building site, then study the architectural theories applicable to the task of thermal design, do comparative analysis of project conditions and historical cases, which finally leads to his choice of specific thermal control mechanisms to be integrated with the formal design of the building.

The above described process has limitations; of which the most important are:
\begin{enumerate}
	\item this traditional process of design is constrained and affected by the architects ability to correctly analyse the data and forecasting the performance of the building under site conditions, making human error a major element of potential failure of the design to meet target performance, 
	\item the ability to test the design is very limited and is not conclusive until the building has been already erected which is not cost effective, 
	\item design alternatives in this case are usually very limited in number due to the designers limitations in terms of cognitive abilities and available time.
\end{enumerate}

This is where the importance of computers comes in, and although computers are heavily used today for the purpose of draughting and virtual modelling, they do not effectively participate in the architectural design thought process. To overcome the limitations described above, the use of computer algorithms and simulation programmes is required; which is the main argument of this thesis.

As an alternative to the conventional design process described above, the use of computers is discussed in this thesis as a tool for generation of architectural solutions for building envelopes, to achieve target thermal performance with passive solar control techniques. The thesis discusses the utilisation of environmental and energy simulation programmes, and computer algorithms; illustrating their current common uses in the architectural design process, and the proposed method of utilisation for the production of thermally efficient designs.

However, as the thesis will illustrate, the procedures for algorithmic design and optimisation are not always simple and straightforward, and some of the steps and choices can be very critical to the success or failure of the whole undertaking, making the careful study, comparison and analysis of possible solutions, a necessity.

\mainmatter

\subimport{./Chapter2/}{Chapter2.tex}
\subimport{./Chapter3/}{Chapter3.tex}
\subimport{./Chapter1/}{Chapter1.tex}
\subimport{./Chapter4/}{Chapter4.tex}
\subimport{./Chapter5/}{Chapter5.tex}

\bibliographystyle{alpha}
\bibliography{./Bibliography}
\end{document}
